\documentclass[a4paper, 10pt, twoside]{article}

\usepackage[top=1in, bottom=1in, left=1in, right=1in]{geometry}
\usepackage[utf8]{inputenc}
\usepackage[spanish, es-ucroman, es-noquoting]{babel}
\usepackage{setspace}
\usepackage{fancyhdr}
\usepackage{lastpage}
\usepackage{amsmath}
\usepackage{amsfonts}
\usepackage{amsthm}
\usepackage{verbatim}
\usepackage{graphicx}
\usepackage{float}
\usepackage{enumitem} % Provee macro \setlist
\usepackage{tabularx}
\usepackage{multirow}
\usepackage{hyperref}
\usepackage{multicol}
\usepackage[toc, page]{appendix}


%%%%%%%%%% Configuración de Fancyhdr - Inicio %%%%%%%%%%
\pagestyle{fancy}
\thispagestyle{fancy}
\lhead{Trabajo Práctico 1 · Ingeniería de Software I}
\rhead{Aboy · Almansi · Canay · Contrufo · Decroix}
\renewcommand{\footrulewidth}{0.4pt}
\cfoot{\thepage /\pageref{LastPage}}

\fancypagestyle{caratula} {
   \fancyhf{}
   \cfoot{\thepage /\pageref{LastPage}}
   \renewcommand{\headrulewidth}{0pt}
   \renewcommand{\footrulewidth}{0pt}
}
%%%%%%%%%% Configuración de Fancyhdr - Fin %%%%%%%%%%


%%%%%%%%%% Miscelánea - Inicio %%%%%%%%%%
% Evita que el documento se estire verticalmente para ocupar el espacio vacío
% en cada página.
\raggedbottom

% Deshabilita sangría en la primer línea de un párrafo.
\setlength{\parindent}{0em}

% Separación entre párrafos.
\setlength{\parskip}{0.5em}

% Separación entre elementos de listas.
\setlist{itemsep=0.5em}

% Asigna la traducción de la palabra 'Appendices'.
\renewcommand{\appendixtocname}{Apéndices}
\renewcommand{\appendixpagename}{Apéndices}
%%%%%%%%%% Miscelánea - Fin %%%%%%%%%%


%%%%%%%%%% Insertar diagrama - Inicio %%%%%%%%%%
\newcommand{\diagramav}[1]{
  \includegraphics[type=png,ext=.png,read=.png,width=16cm]{diagramas/#1}
}

\newcommand{\diagramah}[1]{
  \includegraphics[type=png,ext=.png,read=.png,height=16cm,angle=90]{diagramas/#1}
}
%%%%%%%%%% Insertar diagrama - Fin %%%%%%%%%%

\begin{document}


%%%%%%%%%%%%%%%%%%%%%%%%%%%%%%%%%%%%%%%%%%%%%%%%%%%%%%%%%%%%%%%%%%%%%%%%%%%%%%%
%% Carátula                                                                  %%
%%%%%%%%%%%%%%%%%%%%%%%%%%%%%%%%%%%%%%%%%%%%%%%%%%%%%%%%%%%%%%%%%%%%%%%%%%%%%%%


\thispagestyle{caratula}

\begin{center}

\includegraphics[height=2cm]{DC.png} 
\hfill
\includegraphics[height=2cm]{UBA.jpg} 

\vspace{2cm}

Departamento de Computación,\\
Facultad de Ciencias Exactas y Naturales,\\
Universidad de Buenos Aires

\vspace{4cm}

\begin{Huge}
Trabajo Práctico 1
\end{Huge}

\vspace{0.5cm}

\begin{Large}
Ingeniería de Software I
\end{Large}

\vspace{1cm}

Segundo Cuatrimestre de 2014

\vspace{4cm}

\begin{Large}
Grupo 4
\end{Large}

\vspace{0.5cm}

\begin{tabular}{|c|c|c|}
\hline
Nombre y Apellido & LU & E-mail\\
\hline
Santiago Aboy Solanes & 175/12 & santiaboy2@hotmail.com\\
Emilio Almansi & 674/12 & ealmansi@gmail.com\\
Federico Canay & 250/12 & fcanay@hotmail.com\\
Maximiliano Contrufo & 336/12 & maxicontru@gmail.com\\
Facundo Decroix & 842/11 & fndecroix92@hotmail.com\\
\hline
\end{tabular}

\end{center}

\newpage


%%%%%%%%%%%%%%%%%%%%%%%%%%%%%%%%%%%%%%%%%%%%%%%%%%%%%%%%%%%%%%%%%%%%%%%%%%%%%%%
%% Índice                                                                    %%
%%%%%%%%%%%%%%%%%%%%%%%%%%%%%%%%%%%%%%%%%%%%%%%%%%%%%%%%%%%%%%%%%%%%%%%%%%%%%%%


\tableofcontents

\newpage


%%%%%%%%%%%%%%%%%%%%%%%%%%%%%%%%%%%%%%%%%%%%%%%%%%%%%%%%%%%%%%%%%%%%%%%%%%%%%%%
%% Introducción                                                              %%
%%%%%%%%%%%%%%%%%%%%%%%%%%%%%%%%%%%%%%%%%%%%%%%%%%%%%%%%%%%%%%%%%%%%%%%%%%%%%%%


\section{Introducción}

El CEO de TecnoTaxi nos contactó para solicitarnos crear un sistema. Nuestro objetivo es realizar un sistema para la automatización de las solcitudes de taxis de los pasajeros y la coordinación de los viajes con los taxistas. Antes de realizar el sistema en sí, vamos a proporcionar la mayor cantidad de información posible respecto de cómo nuestra propuesta va a ayudar a cumplir con los objetivos planteados. En particular, vamos a proporcionar un diagrama de contexto y un diagrama de objetivos.

%%%%%%%%%%%%%%%%%%%%%%%%%%%%%%%%%%%%%%%%%%%%%%%%%%%%%%%%%%%%%%%%%%%%%%%%%%%%%%%
%% Desarrollo                                                                %%
%%%%%%%%%%%%%%%%%%%%%%%%%%%%%%%%%%%%%%%%%%%%%%%%%%%%%%%%%%%%%%%%%%%%%%%%%%%%%%%


\section{Desarrollo}


\newpage

%%%%%%%%%%%%%%%%%%%%%%%%%%%%%%%%%%%%%%%%%%%%%%%%%%%%%%%%%%%%%%%%%%%%%%%%%%%%%%%
%% Diagrama de Contexto                                                      %%
%%%%%%%%%%%%%%%%%%%%%%%%%%%%%%%%%%%%%%%%%%%%%%%%%%%%%%%%%%%%%%%%%%%%%%%%%%%%%%%


\section{Diagrama de contexto}

\diagramah{contexto}

A continuación detallaremos algunos fenómenos que creemos que vale la pena aclarar debido a su importancia dentro de la solución, y la brevedad que se les adjudicó en el diagrama.

\newpage

%%%%%%%%%%%%%%%%%%%%%%%%%%%%%%%%%%%%%%%%%%%%%%%%%%%%%%%%%%%%%%%%%%%%%%%%%%%%%%%
%% Modelo de Objetivos                                                       %%
%%%%%%%%%%%%%%%%%%%%%%%%%%%%%%%%%%%%%%%%%%%%%%%%%%%%%%%%%%%%%%%%%%%%%%%%%%%%%%%


\section{Modelo de objetivos}

\newpage

%%%%%%%%%%%%%%%%%%%%%%%%%%%%%%%%%%%%%%%%%%%%%%%%%%%%%%%%%%%%%%%%%%%%%%%%%%%%%%%
%% Escenarios hipotéticos                                                    %%
%%%%%%%%%%%%%%%%%%%%%%%%%%%%%%%%%%%%%%%%%%%%%%%%%%%%%%%%%%%%%%%%%%%%%%%%%%%%%%%


\section{Escenarios hipotéticos}

Los siguientes escenarios ilustran de manera informal situaciones representativas del funcionamiento esperado del sistema.

\subsection{Escenario 1}

Alicia necesita viajar hasta el consultorio de su dentista y elige la moderna empresa TecnoTaxi para solicitar un taxi. Como ella está al día en cuestiones tecnológicas, tiene instalada la aplicación de la empresa en su smartphone y desea realizar la solicitud mediante la misma.

Alicia inicia el proceso de solicitud en la aplicación, ingresando la dirección de origen del viaje, la direccion destino y sus preferencias: un auto moderno. El sistema le muestra mediante la aplicación una lista de opciones de diferentes taxistas que cumplen su pedido y ella elige a Florencia, que maneja un Chevrolet último modelo.

\subsection{Escenario 2}

Omar pidió un taxi para las 17:00hs, para ir a visitar a su suegra. Afortunadamente, ella lo llama para postergar el encuentro. Omar usa su teléfono para avisar a TecnoTaxi y cancelar su viaje.

\subsection{Escenario 3}

Micaela realizó un viaje durante el cual el conductor le realizó múltiples insinuaciones. Ni bien terminó su viaje, dio una calificación de una estrella al conductor y publicó un comentario para que no le pase a nadie más.

\subsection{Escenario 4}

Cada vez que Lucas viaja con su taxista preferido, Roberto, llega a su destino con una sonrisa en su cara. Roberto es un gran comediante amateur. Lucas decide calificar a Roberto positivamente para que todos sepan lo buena persona que es.

\subsection{Escenario 5}

Alfredo desea ir a visitar a su tía, que vive en Boedo. Él es un pasajero usual de TecnoTaxi, por lo que tiene ya descargada en su smartphone la aplicacion móvil de viajes. Al abrir la aplicación, la misma le indica que no se puede encontrar una conexión a internet, por lo que deberá comunicarse por otros medios, sugiriendo la página web de la empresa y el número de teléfono para comunicarse por vía telefónica. Alfredo se dirige entonces a su computadora para solicitar el taxi vía web. Al abrir el navegador web, su página de inicio no carga correctamente. Sorprendido, Alfredo tipea la dirección de la página web indicada por la aplicación, pero su computadora tampoco logra cargarla: no tiene acceso a internet. Indignado por el pésimo servicio de internet que brindan tanto el servidor de internet que utiliza en su casa como el del celular, Alfredo decide hacer su pedido de viaje por teléfono. Llama a la línea telefónica de la empresa de taxis y es atendido por una cordial operadora, y realiza el pedido por este medio.

\subsection{Escenario 6}

Ronaldo va a su masajista todos los martes a las 15:00hs. Necesita un taxi para llegar a horario y no perder la sesión de masajes. Ronaldo llama a TecnoTaxi y programa un viaje de rutina para las 14:30hs de todos los martes. Desde ese momento y hasta el día presente, Ronaldo no perdió ninguna sesión de masajes.

\end{document}